%%%%%%%%%%%%%%%%%%%%%%%%%%%%%%%%%%%%%%%%%
% Keet Colourful Book
% adapted from:
% The Legrand Orange Book
% LaTeX Template
% Version 1 (20/12/25)
%
% This template has been downloaded from:
% tbd
%
% Original author:
% Mathias Legrand (legrand.mathias@gmail.com) with modifications by:
% Vel (vel@latextemplates.com)
% and additional modifications by:
% Keet (info@mkeet.com)
%
% License:
% CC BY-NC-SA 3.0 (http://creativecommons.org/licenses/by-nc-sa/3.0/)
%
% Compiling this template:
% This template uses biber for its bibliography and makeindex for its index.
% When you first open the template, compile it from the command line with the 
% commands below to make sure your LaTeX distribution is configured correctly:
%
% 1) pdflatex main
% 2) makeindex main.idx -s StyleInd.ist
% 3) biber main
% 4) pdflatex main x 2
%
% After this, when you wish to update the bibliography/index use the appropriate
% command above and make sure to compile with pdflatex several times 
% afterwards to propagate your changes to the document.
%
% To get biber to work properly, you have to configure it in your LaTeX editor
% in the "Engine" section, "BibTeX Engine", type "biber" (without quotes)
%
% This template also uses a number of packages which may need to be
% updated to the newest versions for the template to compile. It is strongly
% recommended you update your LaTeX distribution if you have any
% compilation errors.
%
% Important note:
% Chapter heading images should have a 2:1 width:height ratio,
% e.g. 920px width and 460px height.
%
%%%%%%%%%%%%%%%%%%%%%%%%%%%%%%%%%%%%%%%%%

%----------------------------------------------------------------------------------------
%	PACKAGES AND OTHER DOCUMENT CONFIGURATIONS
%----------------------------------------------------------------------------------------

\documentclass[12pt,fleqn]{book} % Default font size and left-justified equations

%----------------------------------------------------------------------------------------
% various packages that may be useful for advanded layout
\usepackage{textcomp} % for additional fonts
%\usepackage[font=small,labelfont=bf]{caption} %for customizable figure and table captions
\usepackage{subcaption} 
% \usepackage{fancyhdr}  % you can choose between 7 different types of heading, besides the default, but it obviously interferes with the template's heading and is thus not recommended
\usepackage{listings}
\usepackage{xspace}
\usepackage{multirow}
\usepackage{multicol} 
\usepackage{lscape}
\usepackage{soul, color} % this with the next one gives convenient colouring options for drafts, e.g. with a \hl{xxx}, the marking becomes ugly green. %\sethlcolor{green}
%\usepackage{url}
%\usepackage{hyperref}
%    \usepackage{endnotes}
%   \let\footnote=\endnote
%\usepackage{xcolor}
\usepackage[many]{tcolorbox}

% if you want to refer to anchors in other documents
%\usepackage{xr-hyper}
%\externaldocument{otherlatexdocument}

 \usepackage[neveradjust]{paralist} %this one is handy if you are short on space or want to have a listing with bulletes, starts etc. e.g. use \begin{compactenum} \item ... \end{comactenum}

%\usepackage[top=3cm, bottom=2.8cm, left=3.1cm, right=3.1cm]{geometry} % example of how to  change page margins, other topics include:
%\usepackage[paperwidth=6in, paperheight=9in]{geometry}

%\usepackage[final,hidelinks]{hyperref} % hidelinks
%\usepackage[final,hidelinks]{hyperref} % 

%\renewcommand{\baselinestretch}{1.02}

\usepackage[font=small,labelfont=it]{caption} %for customizable figure and table captions
\usepackage{subcaption}

%\usepackage{appendix} % assists with numbering the appendix

%------------------------------------------------------------------------------------------

\input{structure} % Insert the commands.tex file which contains the majority of the structure behind the template

\begin{document}
% !BIB TS-program = biber
 !BIB program = biber

%----------------------------------------------------------------------------------------
%	TITLE PAGE
%----------------------------------------------------------------------------------------

\begingroup
\thispagestyle{empty}
\begin{tikzpicture}[remember picture,overlay]
\coordinate [below=10.9cm] (midpoint) at (current page.north);
\node at (current page.north west)
{\begin{tikzpicture}[remember picture,overlay]
\node[anchor=north west,inner sep=0pt] at (0,0) {\includegraphics[width=\paperwidth]{background}}; % Background image PERSONALISE
\draw[anchor=north] (midpoint) node [fill=greenish!30!white,fill opacity=0.6,text opacity=1,inner sep=1.5cm]{\Huge\centering\bfseries\sffamily\parbox[c][][t]{\paperwidth}{\centering Your Title\\ Goes Here \\[15pt] % Book title PERSONALISE. % note that the 'greenish' colour here is adjustable; see structure.tex  
{\Large subtitle or version}\\[20pt] % Subtitle or version or the like PERSONALISE
{\huge Your Name}}}; % Author name PERSONALISE
\end{tikzpicture}};
\end{tikzpicture}
\vfill
\endgroup

%----------------------------------------------------------------------------------------
%	COPYRIGHT PAGE
%----------------------------------------------------------------------------------------

\newpage
~\vfill
\thispagestyle{empty}

\noindent Copyright \textcopyright\ 2025 Your Name\\ % Copyright notice PERSONALISE

\noindent \textsc{Published by Publisher}\\ % Publisher PERSONALISE

\noindent Location: \url{https://mkeet.com/}\\ % URL PERSONALISE OR REMOVE

\noindent Licensed under the Creative Commons Attribution-NonCommercial 4.0 Unported License (the ``License''). You may not use this file except in compliance with the License. You may obtain a copy of the License at \url{https://creativecommons.org/licenses/by-nc/4.0/}. Unless required by applicable law or agreed to in writing, software distributed under the License is distributed on an \textsc{``as is'' basis, without warranties or conditions of any kind}, either express or implied. See the License for the specific language governing permissions and limitations under the License.\\ % License information PERSONALISE

\noindent \textit{First printing, December 2025} % Printing/edition date PERSONALISE

%----------------------------------------------------------------------------------------
%	TABLE OF CONTENTS
%----------------------------------------------------------------------------------------

%\usechapterimagefalse % If you don't want to include a chapter image, use this to toggle images off - it can be enabled later with \usechapterimagetrue

\chapterimage{chapter_head_1.pdf} % Table of contents heading image PERSONALISE

\pagestyle{empty} % No headers

\pagenumbering{roman} %for different numbering style of the front matter part

\tableofcontents % Print the table of contents itself

\cleardoublepage % Forces the first chapter to start on an odd page so it's on the right

\pagestyle{fancy} % Print headers again

\preface

\lipsum[1-3]

\foreword %{Course outline}

\lipsum[1-4]

\clearpage

\pagenumbering{arabic} %to start the numbering of the main content pages
\setcounter{page}{1}

\include{intro} % an introductory chapter that is not part of any main parts

%----------------------------------------------------------------------------------------
%	PART
%----------------------------------------------------------------------------------------

\part{Part One}

%----------------------------------------------------------------------------------------
%	CHAPTER 1
%----------------------------------------------------------------------------------------

\chapterimage{chapter_head_1.pdf} % Chapter heading image PERSONALISE

\chapter{Text Chapter}

There shall always be some text between headings

\section{Paragraphs of Text}\index{Paragraphs of Text}

\lipsum[1-7] % Dummy text

%------------------------------------------------

\section{Preface and foreword}

The command {\tt preface} speaks for itself. This can be adapted, as shown with the {\tt foreword}  command; e.g., to write the course outline if the book bears any related to that or learning paths through the chapters, or rename it in whatever you like. See {\tt structure.tex} ``Customise the foreword section'' comment where to do this.

%------------------------------------------------

\section{Citation}\index{Citation}

This statement requires citation \cite{book_key}; this one is more specific \cite[122]{article_key}. There are many resources on how to cite which sort of entry, as well as tools that automate at least part of it. 

A description that includes practices and fairness in citing related work can also be found here: \url{http://www.meteck.org/teaching/ReferencingWorks.pdf}, which should have received its own citation key and bibentry rather than an inline URL here.

%------------------------------------------------

\section{Lists}\index{Lists}

Lists are useful to present information in a concise and/or ordered way\footnote{Footnote example...}. 

\subsection{Numbered List}\index{Lists!Numbered List}

\begin{enumerate}
\item The first item
\item The second item
\item The third item
\end{enumerate}

\subsection{Bullet Points}\index{Lists!Bullet Points}

\begin{itemize}
\item The first item
\item The second item
\item The third item
\end{itemize}

\subsection{Descriptions and Definitions}\index{Lists!Descriptions and Definitions}

\begin{description}
\item[Name] Description
\item[Word] Definition
\item[Comment] Elaboration
\end{description}

%------------------------------------------------

\section{Learning outcomes}\index{Learning outcomes}

Just in case your book is a textbook and you want define the key learning outcomes per chapter.

\begin{outcome}
This chapter's learning outcomes. 
\end{outcome}

%------------------------------------------------

\section{Sidebar}\index{Sidebar}

A sidebar, or side note, based on the corollary box

\begin{sidenote}[This is an aside]
\lipsum[1-2]
\end{sidenote}

%----------------------------------------------------------------------------------------
%	CHAPTER 2
%----------------------------------------------------------------------------------------

\chapter{In-text Elements}

\section{Theorems}\index{Theorems}

This is an example of theorems.

\subsection{Several equations}\index{Theorems!Several Equations}
This is a theorem consisting of several equations.

\begin{theorem}[Name of the theorem]
In $E=\mathbb{R}^n$ all norms are equivalent. It has the properties:
\begin{align}
& \big| ||\mathbf{x}|| - ||\mathbf{y}|| \big|\leq || \mathbf{x}- \mathbf{y}||\\
&  ||\sum_{i=1}^n\mathbf{x}_i||\leq \sum_{i=1}^n||\mathbf{x}_i||\quad\text{where $n$ is a finite integer}
\end{align}
\end{theorem}

\subsection{Single Line}\index{Theorems!Single Line}
This is a theorem consisting of just one line.

\begin{theorem}
A set $\mathcal{D}(G)$ in dense in $L^2(G)$, $|\cdot|_0$. 
\end{theorem}

%------------------------------------------------

\section{Definitions}\index{Definitions}

This is an example of a definition. A definition could be mathematical or it could define a concept.

\begin{definition}[Definition name]
Given a vector space $E$, a norm on $E$ is an application, denoted $||\cdot||$, $E$ in $\mathbb{R}^+=[0,+\infty[$ such that:
\begin{align}
& ||\mathbf{x}||=0\ \Rightarrow\ \mathbf{x}=\mathbf{0}\\
& ||\lambda \mathbf{x}||=|\lambda|\cdot ||\mathbf{x}||\\
& ||\mathbf{x}+\mathbf{y}||\leq ||\mathbf{x}||+||\mathbf{y}||
\end{align}
\end{definition}

%------------------------------------------------

\section{Notations}\index{Notations}

\begin{notation}
Given an open subset $G$ of $\mathbb{R}^n$, the set of functions $\varphi$ are:
\begin{enumerate}
\item Bounded support $G$;
\item Infinitely differentiable;
\end{enumerate}
a vector space is denoted by $\mathcal{D}(G)$. 
\end{notation}

%------------------------------------------------

\section{Remarks}\index{Remarks}

This is an example of a remark.

\begin{remark}
The concepts presented here are now in conventional employment in mathematics. Vector spaces are taken over the field $\mathbb{K}=\mathbb{R}$, however, established properties are easily extended to $\mathbb{K}=\mathbb{C}$.
\end{remark}

%------------------------------------------------

\section{Corollaries}\index{Corollaries}

This is an example of a corollary.

\begin{corollary}[Corollary name]
The concepts presented here are now in conventional employment in mathematics. Vector spaces are taken over the field $\mathbb{K}=\mathbb{R}$, however, established properties are easily extended to $\mathbb{K}=\mathbb{C}$.
\end{corollary}

%------------------------------------------------

\section{Propositions}\index{Propositions}

This is an example of propositions.

\subsection{Several equations}\index{Propositions!Several Equations}

\begin{proposition}[Proposition name]
It has the properties:
\begin{align}
& \big| ||\mathbf{x}|| - ||\mathbf{y}|| \big|\leq || \mathbf{x}- \mathbf{y}||\\
&  ||\sum_{i=1}^n\mathbf{x}_i||\leq \sum_{i=1}^n||\mathbf{x}_i||\quad\text{where $n$ is a finite integer}
\end{align}
\end{proposition}

\subsection{Single Line}\index{Propositions!Single Line}

\begin{proposition} 
Let $f,g\in L^2(G)$; if $\forall \varphi\in\mathcal{D}(G)$, $(f,\varphi)_0=(g,\varphi)_0$ then $f = g$. 
\end{proposition}

%------------------------------------------------

\section{Examples}\index{Examples}

This is an example of examples.

\subsection{Equation and Text}\index{Examples!Equation and Text}

\begin{example}
Let $G=\{x\in\mathbb{R}^2:|x|<3\}$ and denoted by: $x^0=(1,1)$; consider the function:
\begin{equation}
f(x)=\left\{\begin{aligned} & \mathrm{e}^{|x|} & & \text{si $|x-x^0|\leq 1/2$}\\
& 0 & & \text{si $|x-x^0|> 1/2$}\end{aligned}\right.
\end{equation}
The function $f$ has bounded support, we can take $A=\{x\in\mathbb{R}^2:|x-x^0|\leq 1/2+\epsilon\}$ for all $\epsilon\in\intoo{0}{5/2-\sqrt{2}}$.
\end{example}

\subsection{Paragraph of Text}\index{Examples!Paragraph of Text}

\begin{example}[Example name]
\lipsum[2]
\end{example}

%------------------------------------------------

\section{Review questions, exercises, and answers}\index{Exercises}

This is an example of a review question.

\begin{revq}\label{ex:rq}
This is a good place to ask a review question to test the reader has been reading for meaning.
\end{revq}

This is an example of an exercise.

\begin{exercise}\label{ex:sample}
This is a good place to ask a question to test learning progress or further cement ideas into students' minds.
\end{exercise}

This is an example of an answer.

\begin{answ}
Sample answers can be declared as well, and linked to the specific exercise. 
\end{answ}

There are \LaTeX packages to manage pairing up exercises with answers, which may be a better option.

%------------------------------------------------

\section{Problems}\index{Problems}

\begin{problem}
What is the average airspeed velocity of an unladen swallow?
\end{problem}

%------------------------------------------------

\section{Vocabulary}\index{Vocabulary}

Define a word to improve a students' vocabulary.

\begin{vocabulary}[Word]
Definition of word.
\end{vocabulary}

%------------------------------------------------

\section{Listings}\index{Listings}

Use a listing for some code. The {\tt lstlisting} has a number of parameters for prettier layout than what is shown in Listing~\ref{sampleclass}. 

\begin{lstlisting}[caption={Code snippet of dysfunctional code, copied from {\tt GenerateDummyCode.java}.},label=sampleclass,captionpos=b]
public class GenerateDummyCode {
   public static void main(String[] args) {
      String className = "Eben";
      String newLine = "\n";
      String tab = "\t";
      String classStart = "public class " + className + " {";
      String closeBracket = "}";
      String mainStart = "public static void main(String[] args) {";
      String dummyContent = "";
      String temp = null;
      int size = 10000;
      for (int i = 1; i <= size; i++) {
         temp = String.format("System.out.println(\" %6d .ci satir\");", i);
         dummyContent += tab + tab + temp + newLine;
      }
\end{lstlisting}

%----------------------------------------------------------------------------------------
%	PART
%----------------------------------------------------------------------------------------

\part{Part Two: Short title}

%----------------------------------------------------------------------------------------
%	CHAPTER 3
%----------------------------------------------------------------------------------------

\chapterimage{chapter_head_1.pdf} % Chapter heading image

\chapter{Presenting Information}

\section{Table}\index{Table}

Always start with some text after the section heading, not a table.

\begin{table}[h]
\caption{Table caption always positioned above the table.}
\centering
\begin{tabular}{l l l}
\toprule
\textbf{Treatments} & \textbf{Response 1} & \textbf{Response 2}\\
\midrule
Treatment 1 & 0.0003262 & 0.562 \\
Treatment 2 & 0.0015681 & 0.910 \\
Treatment 3 & 0.0009271 & 0.296 \\
\bottomrule
\end{tabular}
\end{table}

%------------------------------------------------

\section{Figure}\index{Figure}

Always start with some text after the section heading, not an image.

\begin{figure}[h]
\centering\includegraphics[scale=0.5]{placeholder}
\caption{Figure caption always below the figure, or on the side if there's space.}
\end{figure}


%----------------------------------------------------------------------------------------
%	APPENDIX 
%----------------------------------------------------------------------------------------

\appendix

\renewcommand{\thechapter}{\Alph{chapter}}
\renewcommand{\chaptermark}[1]{\markboth{\sffamily\normalsize\bfseries Appendix\space\thechapter.\ #1}{}} % Chapter text font settings
%\begin{appendices}

\part*{Appendix}

\chapter{Adding an appendix}
\label{app:A}

\lipsum[1-4] % Dummy text

%----------------------------------------------------------------------------------------
%	BIBLIOGRAPHY
%----------------------------------------------------------------------------------------

\chapter*{Bibliography}
\addcontentsline{toc}{chapter}{\textcolor{greenish}{Bibliography}} % note that the 'greenish' colour here is adjustable; see structure.tex   
% The entries can be subdivided by type if you so wish:
%\section*{Books}
%\addcontentsline{toc}{section}{Books}
%\printbibliography[heading=bibempty,type=book]
%\section*{Articles}
%\addcontentsline{toc}{section}{Articles}
%\printbibliography[heading=bibempty]
%\printbibliography
\printbibliography[heading=bibliography] 

%%----------------------------------------------------------------------------------------
%%	INDEX
%%----------------------------------------------------------------------------------------
%
\cleardoublepage
\phantomsection
\setlength{\columnsep}{0.75cm}
\addcontentsline{toc}{chapter}{\textcolor{greenish}{Index}} % note that the 'greenish' colour here is adjustable; see structure.tex  
\printindex
%
%----------------------------------------------------------------------------------------

\end{document}
